\chapter{Introduction}
\label{introchap}

Linear algebra calculations are essential in many different fields of science and engineering. Atmospheric science \cite{whalley}, structural engineering \cite{spencer}, quantum physics \cite{vidal} and many other fields heavily rely on linear algebra computations for problem solving and creating simulations. These computations can be very time-consuming depending on the size of the problem. Lowering the time needed to do these computations can significantly enhance the performance of the applications \cite{chen}. It is important to cut down these cost because the size and complexity of scientific computations keep pushing the boundaries of memory and processor technology. Writing computer programs for scientific application is very hard. It is even harder to optimize those programs to improve their performance. Sometimes the performance achieved by the scientific applications is far less than the peak available performance. Many applications achieve 10\% or less of the peak available performance \cite{gropp}.

\paragraph{}
A myriad of software libraries are freely available for linear algebra computations. It is good to have many choices, but it can also become a problem. Sometimes, finding the suitable software package for solving a particular linear algebra problem can be a challenging task. To give a rough idea of how many software libraries are out there, we will briefly discuss the number libraries available for solving various types of linear algebra problems. For basic linear algebra operations, such as matrix-vector, vector-vector and matrix-matrix operations, there are at least twenty-two libraries available \cite{free}. Fifteen libraries are available for higher level operations like solving a dense system of linear equations and finding eigenvalues of a dense matrix using a finite sequence of operations. A dense matrix is a matrix that is primarily populated by nonzero elements. The solvers that uses methods that take a finite sequence of operations to solve a problem are called direct solvers. At least fifteen direct solver packages exist for solving sparse linear systems. A matrix is considered sparse if it is mostly populated by zeros. Twenty-six iterative solver packages are available for solving the sparse systems of linear equations. Iterative solvers use methods that generate a sequence of improving approximate solutions for a class of problems. Iterative methods usually start with an initial guess of the solution and stop when the termination criteria is met. Ten iterative solver packages are available for solving sparse eigenvalue problems. Each of these libraries consist of large numbers of routines, ranging from a few hundreds to well over a thousand. Deciding which software package to use among so many available options and then choosing the appropriate routines from that package to write a program requires a lot of skill and research work.

\paragraph{} 
Some linear algebra problems deal with matrices of enormous sizes. Working with matrices that have hundreds of thousands of entries calls for very powerful computing resources and highly efficient programming and optimization techniques to fully utilize those computing resources. In addition to that, to successfully apply various solvers that are available one has to know which solver is appropriate for what kind of problem. For individuals with little or no computer science and numerical linear algebra background, writing and optimizing programs for solving problems of such huge proportion is extremely challenging, if not impossible. 

\paragraph{}
We have been studying ways to ease the process of creating and using high-performance matrix algebra software. Converting a matrix algebra problem from mere algorithm to highly optimized implementation is a complicated process. First, the programmers have to create efficient implementations from ground up or identify the appropriate numerical routines out of thousands available ones. Then they have to find ways to make these routines run efficiently on the architecture at hand. After that, the process of integrating the routines into a larger application can be very time consuming and difficult. Then the tuning of the routines can be done in various different ways. Three of the most common approaches are: optimizing code fragments manually; using available libraries that have been tuned  for the key numerical algorithms; and, sometimes, performing loop-level code optimizations using compiler-based source transformation tools. At each step of the optimization process, the programmer is confronted with many different of possibilities. To be able to identify and apply the appropriate techniques requires expertise in numerical computation, mathematical software, compilers, and computer architecture.

\paragraph{}
Lighthouse Taxonomy is the product of our attempt to face the daunting challenges of high-performance numerical linear algebra computation. Lighthouse is a guide to the linear system solver routines from LAPACK \cite{lapack} (described in more detail in Chapter 3, Sec. 3.3). We have been working on expanding the Lighthouse framework for the production of matrix algebra software by adding support for sparse matrix algebra computations, integrating high-performance parallel numerical libraries, and automating the process of adding new methods and libraries to the taxonomy. Lighthouse is the first framework that fuses a matrix algebra software collection with code generation and tuning capabilities. Lighthouse produces a high-performance implementation of a matrix algebra problem from its algorithmic description. It provides carefully designed user interfaces to assist users of different backgrounds so that they can easily use the numerical software and code generation and tuning tools.

\paragraph{}
Chapter 2 reviews a number of software libraries that provide support for solving various kind numerical linear algebra problems. It starts by surveying the libraries that offer support routines that is, the routines for performing the most basic level linear algebra operations. In the later sections the dense and sparse direct solver packages are discussed followed by sparse iterative solver packages and libraries that provide preconditioners. Preconditioning is a technique that converts a problem into a form that is more suitable for numerical solution. Finally, it talks about various optimization tools and domain-specific compilers. 

\paragraph{}
Chapter 3 is about the Lighthouse Taxonomy at its current stage. It begins by discussing some of the existing taxonomies and their shortcomings. Then it explains the main objectives and components of the Lighthouse, reviewing the tools used for building the taxonomy and the user interfaces. It discusses in detail the main linear algebra package that Lighthouse includes. Later it explains the user interfaces through which the user can search the dense linear solver routines. It also includes a brief record of the contributions that I have made so far. The chapter ends by explaining the main topic of this thesis which is adding support for solving sparse linear systems to Lighthouse. 

\paragraph{}
Chapter 4 explains the process of adding support for solving sparse linear algebra problems to Lighthouse using a scientific toolkit known as PETSc (Portable, Extensible Toolkit for Scientific Computation). PETSc is a collection of routines and data structures that provide the building blocks for developing parallel numerical solution of partial differential equations (PDEs) and other related problems in high-performance computing. This chapter describes the unique features of PETSc and why we chose to integrate it into Lighthouse. Next, it presents the main use cases to give an idea of how the users will be using PETSc through Lighthouse. Then, it explains how we implemented the use cases and talks about the user interface. Later in the chapter, a summary of the data that has been collected is presented. Next, is reviews the technique that we applied to the data to improve the performance of this new extension to Lighthouse. The chapter ends with a discussion on the results.

\paragraph{}
Chapter 5 provides some concluding remarks and briefly discusses future work.